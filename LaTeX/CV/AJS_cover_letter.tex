%%%%%%%%%%%%%%%%%%%%%%%%%%%%%%%%%%%%%%%%%
% Awesome Cover Letter
% XeLaTeX Template
% Version 1.3 (30/3/2020)
%
% This template has been downloaded from:
% http://www.LaTeXTemplates.com
%
% Original authors:
% Claud D. Park (posquit0.bj@gmail.com)
% Lars Richter (mail@ayeks.de)
% With modifications by:
% Vel (vel@latextemplates.com)
%
% License:
% CC BY-NC-SA 3.0 (http://creativecommons.org/licenses/by-nc-sa/3.0/)
%
% Important note:
% This template must be compiled with XeLaTeX, the below lines will ensure this
%!TEX TS-program = xelatex
%!TEX encoding = UTF-8 Unicode
%
%%%%%%%%%%%%%%%%%%%%%%%%%%%%%%%%%%%%%%%%%

%----------------------------------------------------------------------------------------
%	PACKAGES AND OTHER DOCUMENT CONFIGURATIONS
%----------------------------------------------------------------------------------------

\documentclass[9.5pt, a4paper]{awesome-cv-mod} % A4 paper size by default, use 'letterpaper' for US letter

%\usepackage[activate={true,nocompatibility},final,tracking=true,kerning=true,spacing=true,factor=1100,stretch=10,shrink=10]{microtype}

\usepackage{microtype}

\geometry{left=2.2cm, top=2.0cm, right=2.2cm, bottom=2cm, footskip=.5cm} % Configure page margins with geometry
 
\fontdir[fonts/] % Specify the location of the included fonts

% Color for highlights
\colorlet{awesome}{awesome-red} % Default colors include: awesome-emerald, awesome-skyblue, awesome-red, awesome-pink, awesome-orange, awesome-nephritis, awesome-concrete, awesome-darknight
%\definecolor{awesome}{HTML}{CA63A8} % Uncomment if you would like to specify your own color

% Colors for text - uncomment and modify
%\definecolor{darktext}{HTML}{414141}
%\definecolor{text}{HTML}{414141}
%\definecolor{graytext}{HTML}{414141}
%\definecolor{lighttext}{HTML}{414141}

\renewcommand{\acvHeaderSocialSep}{\quad\textbar\quad} % If you would like to change the social information separator from a pipe (|) to something else

%----------------------------------------------------------------------------------------
%	PERSONAL INFORMATION
%	Comment any of the lines below if they are not required
%----------------------------------------------------------------------------------------

\name{Albertus J.}{Smit}
\address{Biodiversity \& Conservation Biology Department, University of the Western Cape, Bellville, Cape Town, 7535}
\mobile{(+78) 78 300 6005}

\email{ajsmit@uwc.ac.za}
\homepage{www.uwc.ac.za}
\github{https://github.com/ajsmit}

\position{Associate Professor{\enskip\cdotp\enskip}Marine Biology}
\quote{``In the beginning there was nothing, which exploded.'' -- \emph{Terry Pratchett}} % A quote or statement

%----------------------------------------------------------------------------------------
%	RECIPIENT/POSITION/LETTER INFORMATION
%	All of the below lines must be filled out
%----------------------------------------------------------------------------------------

\recipient{The Selection Committee}{Nelson Mandela University} % The company being applied to

\letterdate{\today} % The date on the letter, default is the date of compilation

\lettertitle{Application for the position: Director Institute for Coastal and Marine Research} % The title of the letter

\letteropening{To Whom it may Concern,} % How the letter is opened

\letterclosing{Sincerely,} % How the letter is closed

\letterenclosure[Attached]{Curriculum Vitae with Referrees' names.} % Any enclosures with the letter

\makecvfooter{\today}{A.~J.~Smit~~~·~~~Résumé}{\thepage} % Specify the letter footer with 3 arguments: (<left>, <center>, <right>), leave any of these blank if they are not needed
  
%----------------------------------------------------------------------------------------

\begin{document}

\makecvheader % Print the header

\makelettertitle % Print the title

%----------------------------------------------------------------------------------------
%	LETTER CONTENT
%----------------------------------------------------------------------------------------

\begin{cvletter}

%------------------------------------------------

\lettersection{About Me}

I am an NMU and UCT Alumnus and academic with 25 years of post-Ph.D. national and international academic teaching and research experience. I have a varied and broad academic background in Marine Science, but my research curiosity has become increasingly transdisciplinary. Today my interests straddle the intersection of coastal ecology, oceanography, and the socio-economic wellbeing of coastal communities affected by climate change. My research aligns with the UN SDG11 (“\emph{Sustainable Cities and Communities}”), SDG13 (“\emph{Climate Action}”) and SDG14 (“\emph{Life Below Water}”), ticks many boxes with regards to several of South Africa’s National Strategies, and benefits from the country’s unique Geographical Advantage.

%------------------------------------------------

\lettersection{Why Me?}

I occupy a unique space amongst my academic peers due to my propensity for transgressing traditional discipline boundaries. My strength is an ability to unpack complex problems and address them in the field or \emph{in silico} by drawing on various tools across several academic disciplines. I am known for my creative skill in designing research studies using novel cross-disciplinary approaches, and I draw heavily on my advanced numerical abilities in supporting these endeavours. I instil these same values in postgraduate students.

I have kept abreast with the technological (i.e.~hardware, compute, and software) advancements necessary to use the vast array of open datasets to produce meaningful research outputs locally and globally. Simultaneously, I understand experimental design principles and can comfortably design field campaigns that optimise between the simultaneous constraints of i) robust and representative data harvesting, ii) financial and field resources, and iii) human capital. By forming strategic partnerships between academia and external stakeholders, I ensure the research finds societal relevance and execute it efficiently and effectively.

These skills have allowed me to excel in attracting research grants from national (e.g.~the NRF) and international (e.g.~SANOCEAN, the Belmont Forum, and the European Union) funders. Since 2014 I have brought in ZAR 28.74 million in funding to support my research and students. I work better as part of a team, and I am comfortable leading consortia of international collaborators. 

%------------------------------------------------

\lettersection{Why NMU?}
The opening of the Ocean Sciences Campus has revived NMU’s rich and globally impactful marine research history. The Campus has quickly become a hub for marine science and allied disciplines. Transdisciplinarity is central to the success of the Campus, as seen by the five Research Chairs hosted by the NMU. The Campus also houses SAEON’s Elwandle Node (where I am a Research Associate). This critical mass of institutional knowledge provides a deep foundation to develop a strategic Ocean Sciences vision for the NMU and drive the ICMR’s visibility and relevance in South Africa, the Global South, and the world. A meaningful vision would recognise South Africa’s globally unique geopolitical space—defined by exceptional levels of biodiversity, extreme variability in climates across space and time, a host of impacts stemming from the use of renewable and non‐renewable resources, and an unfortunate diversity of socio‐economic realities of people—as an inspiration for creative transdisciplinary solutions to preserve the functioning of societies and ecosystems in an uncertain climatic future.

I have amassed a valuable array of experience and knowledge during my career. As an academic, I can ensure that this knowledge transfers to a few postgraduate students each year. Alternatively, through the ICMR, I have an opportunity to influence research more broadly. The latter is a compelling reason to apply for this post.

%------------------------------------------------

\end{cvletter}

%----------------------------------------------------------------------------------------

\makeletterclosing % Print the signature and enclosures

\end{document}