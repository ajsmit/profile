%----------------------------------------------------------------------------------------
%	SECTION TITLE
%----------------------------------------------------------------------------------------

\cvsection{Research grants}

%----------------------------------------------------------------------------------------
%	SECTION CONTENT
%----------------------------------------------------------------------------------------

\textbf{Note:} Grant details only provided for the period of my employment at the UWC.

\begin{cventries}

%------------------------------------------------

\cventry
{Belmont Forum} % Funding organization
{Extreme Climatic Events in the Benguela Upwelling System (EXEBUS)} % Grant title
{University of the Western Cape} % Location
{2021--2024} % Date(s)
{ % Description(s)
\begin{cvdescription}
    \item[Funding details] {Funded under the Collaborative Research Action Transdisciplinary Research for Ocean Sustainability.}
    \item[Value] {ZAR 14 million funding.}
    \item[Principal investigator] A.~J.~Smit.
    \item[Co-Principal investigator] Dr. Neville Sweijd.
    \item[Consortium] SADSTIA (South Africa) | JAMSTEC (Japan) | CSIR (South Africa) | Wits (South Africa) | URI (United States of America) | IMR (Norway) | SAWS (South Africa) | UCT (South Africa) | UBC (Canada) | ACCESS (South Africa) | BCC (South Africa, Namibia, and Angola)
    \item[Summary] {
        EXEBUS undertakes an Integrated Ecosystem Assessment (IEA) to establish the roles, trends, and range of variability and the extremities of natural and anthropogenic geophysical, biological, governance, socio- economic features and phenomena, and assess their impact on ecological, sociological, governance, and macroeconomic systems and processes in the Benguela Current Large Marine Ecosystem (BCLME) of South Africa (SA), Namibia, and Angola. The goal is to strengthen the rational basis for management on relevant spatial and temporal scales (up to 2070).}
\end{cvdescription}
}

%------------------------------------------------    

\cventry
{European Union Horizon 2020} % Funding organization
{iAtlantic} % Grant title
{University of Edinburgh} % Location
{2019--2023} % Date(s)
{ % Description(s)
\begin{cvdescription}
    \item[Value] {EUR 14.7 million funding with approx.~ZAR 1 million to the UWC.}
    \item[Principal investigator] Prof. Murray Roberts.
    \item[South East Atlantic Coordinator] A.~J.~Smit.
    \item[Consortium] A consortium of international scientists spanning 34 organisations.
    \item[Summary] {
        Focussing on deep ocean ecosystems, iAtlantic creates a new operational framework that is correctly scaled and integrated to assess marine ecosystem status in an era of multiple stressors. The concept is that by bridging ocean observing systems, exchanging data, researchers, and equipment from South to North and East to West, iAtlantic will predict where and when potentially synergistic effects of global change and multiple stressors will occur.}
\end{cvdescription}
}

%------------------------------------------------    

\cventry
{SANOCEAN} % Funding organization
{Blue Growth Opportunities in Changing Kelp Forests (BlueConnect)} % Grant title
{University of the Western Cape} % Location
{2019--2022} % Date(s)
{ % Description(s)
\begin{cvdescription}
    \item[Funding details] A SA/Norway joint research programme on ocean research (Project Number 287191).
    \item[Value] {~ZAR 4 million to the UWC.}
    \item[Principal investigator] A.~J.~Smit.
    \item[Co-Principal invesigators] Prof.~Thomas Wernberg (Australia) | Dr.~Karen Filbee-Dexter (Norway).
    \item[Consortium] A consortium of scientists in Norway and South Africa.
    \item[Summary] {
        BLUECONNECT, for the first time, creates strategic research partnerships and valuable training opportunities between South Africa and Norway. The research focusses on kelp forests in South Africa, Norway, and globally in the face of climate change. The programme contains a strong capacity building component.
that focus on coastal ecosystem health and sustainable development of marine resources.}
\end{cvdescription}
}

%------------------------------------------------

\cventry
{SANOCEAN} % Funding organization
{Formation, fate and Transport of Microplastics in Marine Coastal Ecosystems (FORTRAN)} % Grant title
{University of the Western Cape} % Location
{2019--2022} % Date(s)
{ % Description(s)
\begin{cvdescription}
    \item[Funding details] A SA/Norway joint research programme on ocean research.
    \item[Value] {~ZAR 3.72 million to the UWC.}
    \item[Principal investigator] A.~J.~Smit.
    \item[Co-Principal invesigators] Prof.~Guven Akdogan (Stellenbosch University) | Drs.~Andy Booth and Lisbet Sorensen (Norway).
    \item[Consortium] A consortium of scientists in Norway and South Africa.
    \item[Summary] {
        FORTRAN addresses existing knowledge gaps related to the fate of plastic litter and micro- and nanoparticles either released into, or formed directly in the marine environment.
that focus on coastal ecosystem health and sustainable development of marine resources.}
\end{cvdescription}
}

%------------------------------------------------

\cventry
{NRF/DSI} % Funding organization
{Extreme Climatic Events in the Coastal Zone} % Grant title
{University of the Western Cape} % Location
{2019--2021} % Date(s)
{ % Description(s)
\begin{cvdescription}
    \item[Funding details] Funded under the NRF Global Change Grand Challenge: Earth System Science Research Programme (Grant UID 118605).
    \item[Value] {ZAR 3.2 million.}
    \item[Principal investigator] A.~J.~Smit.
    \item[Summary] {
        Coupled atmosphere-marine climates affect the ecological/socio-economic important coastal systems. Our aim was to use various sources of data to assess the evidence and potential for change in this zone. This multidisciplinary research took place under two themes, i.e. i) the physical climate dynamics with a view of long-term change, and ii) the biotic (ecosystems and human societies) responses.}
\end{cvdescription}
}

%------------------------------------------------

\cventry
{NRF} % Funding organization
{Upwelling Dynamics in Kelp Beds: Implications for Trophic Function} % Grant title
{University of the Western Cape} % Location
{2018--2020} % Date(s)
{ % Description(s)
\begin{cvdescription}
    \item[Funding details] Funded under the NRF Competitive Programme for Rated Researchers (Grant UID 113350).
    \item[Value] {ZAR 654,000.}
    \item[Principal investigator] A.~J.~Smit.
    \item[Summary] {
        This research focussed on kep forests. We took advantage of the contrasting conditions integrated in the physical ‘climate’ of the nearshore environment in and around kelp forests to address the question: "how do the changing dynamics of the upwelling response influence the ecological structure and function of the kelp ecosystem in the southern Benguela Upwelling System?"}
\end{cvdescription}
}

%------------------------------------------------

\cventry
{NRF} % Funding organization
{Epilithic diatoms on intertidal rocky substrate around the coast of South Africa} % Grant title
{South African Coastal Observation Network} % Location
{2018--2020} % Date(s)
{ % Description(s)
\begin{cvdescription}
    \item[Funding details] Funded under the South Africa-Poland Bilateral Collaboration Project (Grant UID 102283).
    \item[Value] {ZAR 300,000.}
    \item[Principal investigator] Prof.~Thomas Bornman.
    \item[Co-investigator] A.~J.~Smit.
\end{cvdescription}
}

%------------------------------------------------

\cventry
{NRF} % Funding organization
{Thermal Characteristics of the South African Nearshore} % Grant title
{University of the Western Cape} % Location
{2015--2017} % Date(s)
{ % Description(s)
\begin{cvdescription}
    \item[Funding details] Funded under the South Africa-Poland Bilateral Collaboration Project (Grant UID 93609).
    \item[Value] {ZAR 300,000.}
    \item[Principal investigator] A.~J.~Smit.
\end{cvdescription}
}

%------------------------------------------------

\cventry
{NRF} % Funding organization
{Kelps and climate change: South Africa in a global context} % Grant title
{University of the Western Cape} % Location
{2014--2016} % Date(s)
{ % Description(s)
\begin{cvdescription}
    \item[Funding details] Funded under the NRF Competitive Programme for Rated Researchers (Grant UID 87755).
    \item[Value] {ZAR 1.57 million.}
    \item[Principal investigator] A.~J.~Smit.
\end{cvdescription}
}

\end{cventries}